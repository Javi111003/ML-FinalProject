\documentclass[12pt,a4paper]{article}

% Paquetes útiles
\usepackage[utf8]{inputenc}
\usepackage[T1]{fontenc}
\usepackage[spanish]{babel}
\usepackage{hyperref}   % Para hipervínculos en el índice
\usepackage{graphicx}   % Para imágenes
\usepackage{amsmath}    % Para fórmulas matemáticas
\usepackage{cite}       % Para citas en la bibliografía

\begin{document}

% Portada
\begin{titlepage} 
    \centering 
    {\Huge \bfseries Proyecto Final de Aprendizaje de Máquina \par} 
    
    \vspace{0.5cm} 
    
    {\Large Estudio sobre registros de consumo de datos, voz y sms \par} 
    \vfill 
    % empuja el contenido hacia abajo 
    % Autores en esquina inferior izquierda 
    \begin{flushleft} 
        {\large {\textbf{Integrantes:} 

        \vspace{0.5cm} 

        Claudia Hernández Pérez\\ 
        Joel Aparicio Tamayo\\ 
        Kevin Márquez Vega\\ 
        Javier A. González Díaz\\ 
        José Miguel Leyva Cruz\\ 
        Luis E. Amat Cárdenas}} 
    \end{flushleft} 
    
    \vspace{1cm} 
\end{titlepage}


% Índice
\tableofcontents
\newpage

\section{Estudio del estado del arte}

La investigación sobre predicción y clasificación de tráfico en redes ha evolucionado de manera notable en los últimos quince años. En la primera etapa (2010-2013), los modelos estadísticos clásicos como ARMA, ARIMA y SARIMA fueron la base para el análisis de series temporales, mientras que las redes neuronales superficiales (MLP) demostraron capacidad para capturar patrones complejos antes del auge del aprendizaje profundo. 

A partir de 2014, el Deep Learning comenzó a aplicarse con redes profundas (DNN), mostrando mejoras claras frente a los métodos tradicionales. Posteriormente, los modelos híbridos que combinan CNN y LSTM permitieron capturar tanto correlaciones espaciales como temporales, consolidando el enfoque espacio-temporal en la predicción de tráfico. 

Entre 2017 y 2018 surgieron propuestas orientadas a la eficiencia y nuevas representaciones: RCLSTM redujo parámetros manteniendo rendimiento, mientras que la representación del tráfico como imágenes espacio-temporales con CNN 2D mejoró la precisión. En paralelo, se exploraron integraciones más complejas como DMVST-Net, y se mantuvo la relevancia de algoritmos clásicos (SVM, K-NN, árboles) en tareas de clasificación. 

En el periodo 2019-2021 se diversificaron las técnicas, incorporando clustering y reducción de dimensionalidad (K-means, DBSCAN, PCA), además de modelos optimizados como SVR y XGBoost, que superaron a los estadísticos en escenarios no lineales. Las comparativas entre ML y DL confirmaron la superioridad de LSTM, aunque MLP se mantuvo competitivo. 

Los avances recientes (2022-2023) destacan el uso de K-NN con selección de características en LTE, la incorporación de grafos mediante GNN combinados con RNN para modelar dependencias espaciales y temporales, y la validación de modelos estadísticos como SARIMA y Holt-Winters en contextos específicos. También se resaltan alternativas rápidas y robustas como OS-ELM y Random Forest. 

Finalmente, las tendencias emergentes (2024-2025) apuntan hacia arquitecturas modernas como Transformers y TCN, capaces de capturar dependencias globales y temporales con gran precisión. Se exploran enfoques eficientes como HTM y aprendizaje federado para preservar privacidad, y se proponen marcos híbridos adaptativos para entornos 5G. Las revisiones recientes concluyen que el aprendizaje profundo moderno (LSTM, GNN, Transformers) domina en problemas espacio-temporales complejos, mientras que los métodos clásicos siguen siendo útiles en escenarios simples o con restricciones de recursos.

\section{Estudio sobre los datos}

La Empresa de Telecomunicaciones de Cuba S.A. (ETECSA) proporcionó un conjunto de datos con el propósito de desarrollar estudios y modelos basados en técnicas de Aprendizaje de Máquina (Machine Learning). Dichos datos reflejan el uso de diversos servicios de telecomunicaciones por parte de los usuarios, tales como llamadas telefónicas, mensajes de texto (SMS), recargas de saldo y consumo de datos móviles. 

El objetivo principal de este análisis es comprender la estructura, el contenido y las características de los datos, con vistas a su preparación y posterior aplicación en modelos predictivos o de análisis de comportamiento.

\subsection{Descripción general del dataset}
El conjunto de datos se encuentra en formato tabular y contiene aproximadamente 10 mil registros y 40 variables, distribuidas en columnas que describen los diferentes aspectos de cada transacción o evento de uso de servicios. 

Cada fila representa un registro detallado de uso de servicio (CDR, por sus siglas en inglés: Call Detail Record), que documenta información relacionada con un evento generado por el cliente, como una llamada, el envío de un mensaje o una conexión a internet móvil.

A continuación, se presenta un resumen de los tipos de variables más relevantes:

\begin{center}
\begin{tabular}{|p{3cm}|p{5cm}|p{7.5cm}|}
\hline
\textbf{Tipo de variable} & \textbf{Ejemplo de campos} & \textbf{Descripción general} \\
\hline
Identificadores & CDR\_ID, OBJ\_ID, OWNER\_CUST\_ID & Identifican de manera única cada registro, objeto o cliente asociado. \\
\hline
Temporales & START\_DATE, END\_DATE & Indican la fecha y hora de inicio y fin del servicio utilizado. \\
\hline
Categóricas & SERVICE\_CATEGORY, FLOW\_TYPE, USAGE\_SERVICE\_TYPE & Especifican el tipo de servicio, su categoría (voz, datos, SMS, recarga) y dirección del tráfico (entrante/saliente). \\
\hline
Numéricas & ACTUAL\_USAGE, ACTUAL\_CHARGE, TOTAL\_TAX\_AMOUNT & Miden el volumen de uso (minutos, megabytes, mensajes) y los cargos monetarios asociados. \\
\hline
Listas o estructuras anidadas & CHARGE\_LIST, CHARGE\_SERVICE\_INFO, BALANCE\_CHG\_LIST & Detallan cargos, impuestos y modificaciones de saldo que se producen en cada evento. \\
\hline
\end{tabular}
\end{center}

Estos campos se complementan con información auxiliar relacionada con unidades de medida, identificadores de cuenta, ciclos de facturación y valores reservados para futuras ampliaciones del sistema.

\subsection{Distribución de los datos}


\section{Detección de patrones}
Explicación de los métodos usados para encontrar patrones en los datos.

\section{Reducción de dimensionalidad}
Discusión sobre técnicas como PCA, t-SNE, etc.

\section{AutoML}
Descripción de herramientas y resultados obtenidos con AutoML.

% Ejemplo de bibliografía manual
\begin{thebibliography}{99}
\bibitem{ejemplo1} Autor, Título del libro/artículo, Editorial, Año.
\bibitem{ejemplo2} Autor, Título del paper, Revista, Volumen, Año.
\end{thebibliography}

% Si prefieres usar BibTeX:
% \bibliographystyle{plain}
% \bibliography{referencias}

\end{document}