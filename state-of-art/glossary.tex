\documentclass[11pt,a4paper]{article}

\usepackage[spanish]{babel}
\usepackage[utf8]{inputenc}
\usepackage[T1]{fontenc}
\usepackage{geometry}
\usepackage{setspace}
\usepackage{hyperref}

\geometry{margin=2.5cm}
\onehalfspacing

\title{Glosario de Términos y Siglas\\
Predicción de Tráfico en Redes mediante Machine Learning}
\author{}
\date{}

\begin{document}

\maketitle

\section{Glosario}

\begin{description}

\item[ML (Machine Learning)]  
Conjunto de métodos que permiten a un sistema aprender patrones a partir de datos sin programación explícita.

\item[DL (Deep Learning)]  
Subcampo de Machine Learning basado en redes neuronales profundas con múltiples capas ocultas.

\item[ML clásico]  
Algoritmos de Machine Learning no profundos, como k-NN, SVM, Árboles de decisión y modelos estadísticos.

\item[DL avanzado]  
Arquitecturas profundas modernas como Transformers, GNNs y modelos híbridos.

\item[ARMA (AutoRegressive Moving Average)]  
Modelo estadístico para series temporales que combina componentes autorregresivos y de media móvil.

\item[ARIMA (AutoRegressive Integrated Moving Average)]  
Extensión de ARMA que incorpora diferenciación para manejar series no estacionarias.

\item[SARIMA (Seasonal ARIMA)]  
Modelo ARIMA que incluye explícitamente componentes estacionales.

\item[Holt-Winters]  
Método de suavizado exponencial que modela nivel, tendencia y estacionalidad.

\item[Prophet]  
Modelo aditivo para series temporales desarrollado por Meta, robusto ante estacionalidades múltiples.

\item[MLP (Multi-Layer Perceptron)]  
Red neuronal feed-forward clásica compuesta por múltiples capas totalmente conectadas.

\item[DNN (Deep Neural Network)]  
Red neuronal profunda con varias capas ocultas.

\item[RNN (Recurrent Neural Network)]  
Red neuronal diseñada para procesar datos secuenciales.

\item[LSTM (Long Short-Term Memory)]  
Tipo de RNN capaz de capturar dependencias de largo plazo en series temporales.

\item[GRU (Gated Recurrent Unit)]  
Variante simplificada de LSTM con menor número de parámetros.

\item[RCLSTM]  
LSTM de complejidad reducida o dispersa que mantiene rendimiento con menos parámetros.

\item[CNN (Convolutional Neural Network)]  
Red neuronal especializada en la extracción de patrones espaciales.

\item[CNN 2D]  
CNN aplicada a representaciones matriciales espacio-temporales del tráfico.

\item[CNN + LSTM]  
Modelo híbrido donde CNN captura correlaciones espaciales y LSTM dependencias temporales.

\item[GNN (Graph Neural Network)]  
Red neuronal diseñada para datos estructurados como grafos.

\item[GCN (Graph Convolutional Network)]  
Tipo de GNN que generaliza la operación de convolución a grafos.

\item[Transformer]  
Arquitectura basada en mecanismos de atención que captura dependencias globales de largo alcance.

\item[TCN (Temporal Convolutional Network)]  
Red convolucional causal diseñada para modelado temporal.

\item[k-NN (k-Nearest Neighbors)]  
Algoritmo de aprendizaje basado en instancias que utiliza vecinos más cercanos.

\item[SVM (Support Vector Machine)]  
Algoritmo de clasificación y regresión basado en maximización del margen.

\item[SVR (Support Vector Regression)]  
Versión de SVM para tareas de regresión.

\item[SOS-vSVR]  
SVR optimizado mediante técnicas de optimización por enjambre.

\item[Random Forest]  
Método de ensamble basado en múltiples árboles de decisión.

\item[XGBoost]  
Algoritmo de gradient boosting eficiente y altamente competitivo.

\item[Naive Bayes]  
Clasificador probabilístico basado en el teorema de Bayes con independencia condicional.

\item[K-means]  
Algoritmo de clustering no supervisado basado en centroides.

\item[DBSCAN]  
Algoritmo de clustering basado en densidad capaz de detectar ruido y anomalías.

\item[PCA (Principal Component Analysis)]  
Técnica de reducción de dimensionalidad basada en varianza.

\item[ELM (Extreme Learning Machine)]  
Red neuronal de una sola capa con entrenamiento extremadamente rápido.

\item[OS-ELM]  
Variante de ELM para aprendizaje secuencial u online.

\item[HTM (Hierarchical Temporal Memory)]  
Modelo inspirado en la neocorteza para procesamiento temporal, no basado en deep learning.

\item[Throughput]  
Tasa efectiva de transmisión de datos en una red.

\item[LTE (Long Term Evolution)]  
Estándar de comunicaciones móviles de cuarta generación (4G).

\item[5G / 6G]  
Quinta y futura sexta generación de redes móviles.

\item[QoS (Quality of Service)]  
Conjunto de métricas que miden la calidad de una red (latencia, pérdida, ancho de banda).

\item[Tráfico espacio-temporal]  
Modelado del tráfico considerando dimensiones espaciales y temporales.

\item[Topología de red]  
Estructura de interconexión entre los nodos de una red.

\item[FL (Federated Learning)]  
Paradigma de aprendizaje distribuido que preserva la privacidad de los datos.

\item[Aprendizaje distribuido]  
Entrenamiento de modelos en múltiples nodos computacionales.

\item[MAPE (Mean Absolute Percentage Error)]  
Métrica de error basada en el porcentaje medio absoluto.

\item[Baseline]  
Modelo de referencia utilizado para comparación de rendimiento.

\end{description}

\end{document}