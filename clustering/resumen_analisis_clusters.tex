\documentclass[11pt, a4paper]{article}
\usepackage{graphicx}
\usepackage{booktabs}
\usepackage{array}
\usepackage{multirow}
\usepackage{xcolor}
\usepackage{hyperref}
\usepackage{float}
\usepackage{geometry}
\usepackage{amsmath}
\usepackage{amssymb}

\geometry{margin=1in}

% Definición de colores para los clusters
\definecolor{altoregular}{RGB}{31, 119, 180}
\definecolor{bajoextremo}{RGB}{255, 127, 14}
\definecolor{altoextremo}{RGB}{214, 39, 40}
\definecolor{bajoregular}{RGB}{44, 160, 44}

\title{\textbf{Resumen Completo de Análisis de Clusters} \\[0.5em] \large Estudio de Perfiles de Consumo de Usuarios}
\author{ML Final Project}
\date{\today}

\begin{document}

\maketitle

\tableofcontents
\newpage

\section{Introducción y Contexto}

Este documento presenta un resumen ejecutivo de todos los análisis estadísticos y descriptivos realizados sobre los clusters de usuarios identificados mediante algoritmos de aprendizaje no supervisado. El objetivo es comprender los patrones de consumo de datos, SMS y voz en diferentes segmentos de población de usuarios.

\section{Metodología de Clustering}

\subsection{Algoritmo Utilizado}

Se empleó el algoritmo \textbf{K-Means} con el siguiente procedimiento:

\begin{enumerate}
    \item \textbf{Preparación de datos}: Normalización con StandardScaler
    \item \textbf{Selección de K}: Análisis de Silhouette para k = 3, 4, 5, 6, 7, 8
    \item \textbf{Resultado}: K = 4 seleccionado como óptimo
\end{enumerate}

\subsection{Justificación de K=4}

\begin{table}[H]
\centering
\caption{Comparación de Silhouette Score por número de clusters}
\begin{tabular}{ccccc}
\toprule
K & Silhouette Score & Singletons & Interpretabilidad & Selección \\
\midrule
3 & 0.4739 & Alto & Media & \\
4 & 0.4901 & Bajo & \textbf{Excelente} & \checkmark \\
5 & 0.4902 & Muy Alto & Baja & \\
6+ & $<0.48$ & Extremo & Pobre & \\
\bottomrule
\end{tabular}
\end{table}

K=4 fue seleccionado por:
\begin{itemize}
    \item Silhouette Score máximo (0.4901)
    \item Menor número de singletons respecto a K=5
    \item Mejor interpretabilidad de perfiles
    \item Balance óptimo entre granularidad y estabilidad
\end{itemize}

\section{Descripción de los Clusters}

Los 4 clusters identificados representan perfiles de consumo distintos:

\subsection{Cluster 1: Alto Regular}
\begin{itemize}
    \item \textbf{Tamaño}: 396 usuarios (44.9\%)
    \item \textbf{Consumo promedio}: 111.8 MB
    \item \textbf{Rango}: 113 KB a 2,710 MB
    \item \textbf{Características}: Heavy users con consumo consistente y predecible
    \item \textbf{Color}: \textcolor{altoregular}{\textbullet\ Azul}
\end{itemize}

\subsection{Cluster 2: Bajo Extremo}
\begin{itemize}
    \item \textbf{Tamaño}: 444 usuarios (50.3\%)
    \item \textbf{Consumo promedio}: 53.2 MB
    \item \textbf{Rango}: 60 bytes a 1,335 MB
    \item \textbf{Características}: Light users ocasionales con comportamiento variable
    \item \textbf{Color}: \textcolor{bajoextremo}{\textbullet\ Naranja}
\end{itemize}

\subsection{Cluster 3: Alto Extremo}
\begin{itemize}
    \item \textbf{Tamaño}: 1 usuario (0.1\%)
    \item \textbf{Consumo total}: 14,335.7 GB
    \item \textbf{Características}: Power user con consumo extremadamente alto
    \item \textbf{Color}: \textcolor{altoextremo}{\textbullet\ Rojo}
\end{itemize}

\subsection{Cluster 4: Bajo Regular}
\begin{itemize}
    \item \textbf{Tamaño}: 41 usuarios (4.6\%)
    \item \textbf{Consumo promedio}: 62.3 MB
    \item \textbf{Rango}: 827 KB a 644 MB
    \item \textbf{Características}: Light users regulares con patrones predecibles
    \item \textbf{Color}: \textcolor{bajoregular}{\textbullet\ Verde}
\end{itemize}

\newpage

\section{Análisis Estadístico de Perfiles}

\subsection{Estadística Descriptiva}

\begin{table}[H]
\centering
\caption{Resumen estadístico por cluster}
\footnotesize
\begin{tabular}{lccccc}
\toprule
\textbf{Cluster} & \textbf{n} & \textbf{Media (MB)} & \textbf{Mediana (MB)} & \textbf{Máx (MB)} \\
\midrule
\textcolor{altoregular}{Alto Regular} & 396 & 111.81 & 16.08 & 2,710.55 \\
\textcolor{bajoextremo}{Bajo Extremo} & 444 & 53.23 & 6.17 & 1,335.74 \\
\textcolor{bajoregular}{Bajo Regular} & 41 & 62.26 & 27.85 & 644.84 \\
\textcolor{altoextremo}{Alto Extremo} & 1 & 14,335.65 & 14,335.65 & 14,335.65 \\
\bottomrule
\end{tabular}
\end{table}

\subsection{Variabilidad (Coeficiente de Variación)}

El Coeficiente de Variación (CV) mide la predictibilidad del comportamiento:

\begin{table}[H]
\centering
\caption{Coeficiente de Variación (CV) por cluster}
\begin{tabular}{lrrr}
\toprule
\textbf{Cluster} & \textbf{CV Consumo} & \textbf{CV Duración} & \textbf{CV Sesiones} \\
\midrule
\textcolor{altoregular}{Alto Regular} & 263.8\% & 25.0\% & 55.8\% \\
\textcolor{bajoextremo}{Bajo Extremo} & 236.6\% & 89.7\% & 83.3\% \\
\textcolor{bajoregular}{Bajo Regular} & 185.3\% & 61.8\% & 62.8\% \\
\bottomrule
\end{tabular}

\textit{Nota: CV Alto Extremo no calculable (n=1)}
\end{table}

\subsubsection{Interpretación de CV}

\begin{itemize}
    \item \textbf{CV < 50\%}: Comportamiento muy predecible
    \item \textbf{50\% $\leq$ CV < 100\%}: Comportamiento moderadamente predecible
    \item \textbf{CV $\geq$ 100\%}: Comportamiento muy variable e impredecible
\end{itemize}

\textbf{Hallazgos}:
\begin{itemize}
    \item \textcolor{bajoregular}{Bajo Regular} es el más predecible (CV=185.3\% en consumo)
    \item \textcolor{altoregular}{Alto Regular} muestra duración muy predecible (CV=25.0\%)
    \item \textcolor{bajoextremo}{Bajo Extremo} es el menos predecible (CV=89.7\% en duración)
\end{itemize}

\subsection{Tests de Significancia Estadística}

Se realizaron tests para verificar si las diferencias entre clusters son estadísticamente significativas:

\subsubsection{ANOVA (Análisis de Varianza)}

\begin{table}[H]
\centering
\caption{Resultados ANOVA}
\begin{tabular}{lccc}
\toprule
\textbf{Variable} & \textbf{F-statistic} & \textbf{p-value} & \textbf{Significancia} \\
\midrule
Consumo Total & 1,422.61 & $<0.0001$ & *** \\
Duración & 742.63 & $<0.0001$ & *** \\
\bottomrule
\end{tabular}
\end{table}

Ambos tests son \textbf{altamente significativos} ($p < 0.0001$), indicando que los clusters difieren genuinamente en consumo y duración.

\subsubsection{Kruskal-Wallis (Test No-Paramétrico)}

\begin{align*}
H &= 54.25 \\
p &< 0.0001 \text{ (***)} \\
\text{Conclusión}: &\text{ Diferencias significativas sin asumir normalidad}
\end{align*}

\subsection{Correlaciones Duración-Consumo}

Análisis de la relación entre duración de sesiones y consumo de datos:

\begin{table}[H]
\centering
\caption{Correlaciones por cluster (Pearson y Spearman)}
\footnotesize
\begin{tabular}{lccccc}
\toprule
\textbf{Cluster} & \textbf{n} & \textbf{Pearson r} & \textbf{p-value} & \textbf{Spearman rho} & \textbf{p-value} \\
\midrule
\textcolor{bajoextremo}{Bajo Extremo} & 444 & 0.235 & 0.000001 & 0.463 & 0.000000 \\
\textcolor{bajoregular}{Bajo Regular} & 41 & 0.009 & 0.954 & -0.067 & 0.677 \\
\textcolor{altoregular}{Alto Regular} & 396 & 0.034 & 0.498 & 0.168 & 0.001 \\
\bottomrule
\end{tabular}
\end{table}

\textbf{Interpretación}:
\begin{itemize}
    \item \textcolor{bajoextremo}{Bajo Extremo}: Correlación moderada (r=0.235, significativa), lo que indica que mayor duración se asocia con mayor consumo en usuarios ocasionales
    \item \textcolor{altoregular}{Alto Regular}: Correlación débil (r=0.034, no significativa en Pearson), pero significativa en Spearman (rho=0.168), sugeriendo relación no lineal
    \item \textcolor{bajoregular}{Bajo Regular}: Sin correlación (r=0.009), duración no predice consumo
\end{itemize}

\newpage

\section{Análisis de Patrones por Tipo de Servicio}

\subsection{Distribución por Tipo de Consumo (Datos, SMS, Voz)}

\subsubsection{Consumo de Datos}

\begin{table}[H]
\centering
\caption{Perfil de consumo de datos por cluster}
\begin{tabular}{lrrr}
\toprule
\textbf{Cluster} & \textbf{Promedio (MB)} & \textbf{Mediana (MB)} & \textbf{Desviación (MB)} \\
\midrule
\textcolor{altoregular}{Alto Regular} & 72.45 & 8.53 & 259.93 \\
\textcolor{bajoextremo}{Bajo Extremo} & 36.18 & 3.82 & 85.74 \\
\textcolor{bajoregular}{Bajo Regular} & 41.70 & 17.62 & 80.24 \\
\bottomrule
\end{tabular}
\end{table}

\textbf{Hallazgo}: Alto Regular consume 2x más datos que los demás clusters.

\subsubsection{Consumo de SMS}

\begin{table}[H]
\centering
\caption{Distribución de SMS por cluster}
\begin{tabular}{lrr}
\toprule
\textbf{Cluster} & \textbf{Usuarios SMS} & \textbf{Porcentaje} \\
\midrule
\textcolor{altoregular}{Alto Regular} & 177 & 44.7\% \\
\textcolor{bajoextremo}{Bajo Extremo} & 245 & 55.2\% \\
\textcolor{bajoregular}{Bajo Regular} & 32 & 78.0\% \\
\bottomrule
\end{tabular}
\end{table}

\textbf{Hallazgo}: Bajo Regular es más propenso a usar SMS (78\% vs 45\% en Alto Regular).

\subsubsection{Consumo de Voz}

\begin{table}[H]
\centering
\caption{Distribución de Voz por cluster}
\begin{tabular}{lrr}
\toprule
\textbf{Cluster} & \textbf{Usuarios Voz} & \textbf{Porcentaje} \\
\midrule
\textcolor{altoregular}{Alto Regular} & 148 & 37.4\% \\
\textcolor{bajoextremo}{Bajo Extremo} & 134 & 30.2\% \\
\textcolor{bajoregular}{Bajo Regular} & 18 & 43.9\% \\
\bottomrule
\end{tabular}
\end{table}

\textbf{Hallazgo}: Distribución relativamente balanceada entre clusters, con Bajo Regular ligeramente más propenso a usar voz.

\newpage

\section{Análisis Temporal: Influencia por Horario}

\subsection{Dominancia Temporal}

Análisis de qué cluster domina (mayor consumo) en cada hora del período analizado:

\begin{table}[H]
\centering
\caption{Dominancia de clusters por hora del día}
\begin{tabular}{ccccc}
\toprule
\textbf{Hora UTC} & \textbf{Cluster Dominante} & \textbf{Consumo (GB)} & \textbf{\% Dominancia} & \textbf{Estabilidad} \\
\midrule
00:00 & \textcolor{altoregular}{Alto Regular} & 14.66 & 61.0\% & Alta \\
01:00 & \textcolor{altoregular}{Alto Regular} & 6.49 & 57.3\% & Alta \\
02:00 & \textcolor{altoregular}{Alto Regular} & 8.86 & 66.8\% & Muy Alta \\
03:00 & \textcolor{altoextremo}{Alto Extremo} & 14.11 & 60.0\% & Baja \\
04:00 & \textcolor{altoregular}{Alto Regular} & 3.43 & 59.8\% & Alta \\
05:00 & \textcolor{altoregular}{Alto Regular} & 2.31 & 53.3\% & Media \\
06:00 & \textcolor{bajoextremo}{Bajo Extremo} & 1.36 & 59.0\% & Media \\
07:00 & \textcolor{altoregular}{Alto Regular} & 0.11 & 42.7\% & Baja \\
\bottomrule
\end{tabular}
\end{table}

\subsection{Contribución Total al Consumo}

\begin{table}[H]
\centering
\caption{Aportación de cada cluster al consumo total}
\begin{tabular}{lrr}
\toprule
\textbf{Cluster} & \textbf{Consumo Total (GB)} & \textbf{Porcentaje} \\
\midrule
\textcolor{altoregular}{Alto Regular} & 44.28 & 52.2\% \\
\textcolor{bajoextremo}{Bajo Extremo} & 23.63 & 27.9\% \\
\textcolor{altoextremo}{Alto Extremo} & 14.34 & 16.9\% \\
\textcolor{bajoregular}{Bajo Regular} & 2.55 & 3.0\% \\
\midrule
\textbf{TOTAL} & \textbf{84.80 GB} & \textbf{100.0\%} \\
\bottomrule
\end{tabular}
\end{table}

\subsection{Matriz de Influencia Relativa (\%)}

Porcentaje de consumo de cada cluster en cada hora:

\begin{table}[H]
\centering
\caption{Influencia relativa de clusters por hora UTC (\%)}
\footnotesize
\begin{tabular}{lrrrr}
\toprule
\textbf{Hora UTC} & \textcolor{altoregular}{\textbf{Alto Regular}} & \textcolor{bajoextremo}{\textbf{Bajo Extremo}} & \textcolor{altoextremo}{\textbf{Alto Extremo}} & \textcolor{bajoregular}{\textbf{Bajo Regular}} \\
\midrule
00:00 & 61.0\% & 36.1\% & 0.0\% & 3.0\% \\
01:00 & 57.3\% & 41.8\% & 0.0\% & 0.9\% \\
02:00 & 66.8\% & 29.1\% & 1.7\% & 2.3\% \\
03:00 & 32.8\% & 6.0\% & 60.0\% & 1.3\% \\
04:00 & 59.8\% & 32.9\% & 0.0\% & 7.3\% \\
05:00 & 53.3\% & 37.1\% & 0.0\% & 9.6\% \\
06:00 & 30.0\% & 59.0\% & 0.0\% & 11.0\% \\
07:00 & 42.7\% & 40.2\% & 0.0\% & 17.1\% \\
\bottomrule
\end{tabular}
\end{table}

\subsection{Patrones Horarios Específicos por Cluster}

\subsubsection{\textcolor{altoregular}{Alto Regular} - Variabilidad Media}

\begin{itemize}
    \item \textbf{Patrón}: Consistente con picos matutinos
    \item \textbf{Máximo}: 00:00 UTC (14.7 GB)
    \item \textbf{Mínimo}: 07:00 UTC (0.11 GB)
    \item \textbf{Coeficiente de Variación}: $\approx 50\%$ (Predecible)
    \item \textbf{Interpretación}: Heavy users con consumo regulado y predecible
    \item \textbf{Recomendación Operacional}: Provisionar recursos en 00:00-02:00 UTC; optimizar para streaming
\end{itemize}

\subsubsection{\textcolor{bajoextremo}{Bajo Extremo} - Variabilidad Alta}

\begin{itemize}
    \item \textbf{Patrón}: Crece hacia final del período
    \item \textbf{Máximo}: 01:00 UTC (9.3 GB)
    \item \textbf{Mínimo}: 03:00 UTC (1.4 GB)
    \item \textbf{Coeficiente de Variación}: $\approx 100\%$ (Impredecible)
    \item \textbf{Interpretación}: Usuarios ocasionales con comportamiento variable
    \item \textbf{Recomendación Operacional}: Paquetes de datos flexibles; monitoreo proactivo
\end{itemize}

\subsubsection{\textcolor{altoextremo}{Alto Extremo} - Variabilidad Extrema}

\begin{itemize}
    \item \textbf{Patrón}: Un único pico en 03:00 UTC
    \item \textbf{Máximo}: 03:00 UTC (14.1 GB $\approx$ 60\% del consumo en esa hora)
    \item \textbf{Resto del período}: Casi cero
    \item \textbf{Coeficiente de Variación}: Máximo (Muy impredecible)
    \item \textbf{Interpretación}: Power user (1 usuario) con actividad extremadamente localizada
    \item \textbf{Recomendación Operacional}: QoS separado; traffic shaping; tarifa especial
\end{itemize}

\subsubsection{\textcolor{bajoregular}{Bajo Regular} - Variabilidad Moderada}

\begin{itemize}
    \item \textbf{Patrón}: Crecimiento progresivo
    \item \textbf{Máximo}: 07:00 UTC (0.44 GB)
    \item \textbf{Mínimo}: 03:00 UTC (0.03 GB)
    \item \textbf{Coeficiente de Variación}: $\approx 70\%$ (Variable pero predecible)
    \item \textbf{Interpretación}: Usuarios light que se activan gradualmente
    \item \textbf{Recomendación Operacional}: Fomentar crecimiento; planes graduales; campañas en horas pico
\end{itemize}

\newpage


\section{Hallazgos Clave}

\begin{enumerate}
    \item \textbf{Dominancia de Alto Regular}: Este cluster aporta más del 52\% del consumo total y domina el 75\% del período (6 de 8 horas), siendo crítico para la operación de red.
    
    \item \textbf{Anomalía en 03:00 UTC}: Alto Extremo (1 solo usuario) domina completamente en 03:00 UTC, consumiendo el 60\% del tráfico en esa hora. Requiere atención especial.
    
    \item \textbf{Transición en 06:00 UTC}: Bajo Extremo supera a Alto Regular por primera vez en 06:00 UTC, indicando cambio de patrón hacia final del período.
    
    \item \textbf{Predictibilidad}: Alto Regular es más predecible (CV bajo en duración), mientras que Bajo Extremo es altamente impredecible (CV alto).
    
    \item \textbf{Correlación Débil Duración-Consumo}: En Alto Regular, la duración no es buen predictor de consumo, sugiriendo que otros factores (tipos de aplicación, calidad de video, etc.) son más importantes.
\end{enumerate}

\section{Conclusiones}

Los análisis estadísticos realizados validan la existencia de 4 clusters distintivos y estables de usuarios, cada uno con características únicas de consumo y patrones temporales específicos. Los clusters no son artefactos aleatorios, sino segmentos reales diferenciados estadísticamente (ANOVA p<0.0001, Kruskal-Wallis p<0.0001).

El cluster \textbf{Alto Regular} es el segmento más importante estratégicamente, seguido por \textbf{Bajo Extremo}. La presencia de un único usuario en \textbf{Alto Extremo} requiere monitoreo especial.

Las recomendaciones operacionales deben adaptarse al patrón temporal específico de cada cluster, con provisioning diferenciado y QoS segregado para optimizar tanto la experiencia de usuario como la eficiencia operacional.

\end{document}
