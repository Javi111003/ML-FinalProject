\documentclass[a4paper,12pt]{article}
\usepackage[spanish]{babel}
\usepackage[utf8]{inputenc}
\usepackage[T1]{fontenc}
\usepackage{hyperref}
\usepackage{graphicx}
\usepackage{booktabs}
\usepackage{longtable}
\usepackage{geometry}
\usepackage{amsmath}
\usepackage{xcolor}
\usepackage{enumitem}
\geometry{margin=2.5cm}

% Configurar hyperref
\hypersetup{
    colorlinks=true,
    linkcolor=blue,
    filecolor=magenta,
    urlcolor=cyan,
    pdftitle={Análisis de Clustering},
    pdfauthor={Autor}
}

\begin{document}

\section*{Análisis de clustering}

Se presentan varios análisis del conjunto de datos usando clustering para agrupar usuarios.

\subsection*{Objetivos}

\begin{itemize}[leftmargin=*]
    \item Separar los consumidores en altos consumidores, consumidores promedio y bajos consumidores para realizar tratamientos especializados.
    \item Encontrar patrones horarios para cada grupo para posibles estudios sobre ampliación de ancho de banda en ciertos horarios o mayor velocidad para cierto tipo de usuarios.
    \item Estudiar consumo vs tiempo de actividad en cada grupo para ver si es más ventajoso proponer un plan por tiempo y no por consumo (ej. nauta hogar).
    \item Analizar si los bajos consumidores de datos prefieren otros servicios (se encuentran entre los consumidores promedio o altos de otros servicios).
    \item Evaluar y optimizar el número de clusters usando Coeficiente de Silueta para determinar la segmentación óptima de usuarios (K=2 a 10).
\end{itemize}

\subsection*{Resultados}

\subsubsection*{Objetivo 1: Separación de consumidores en perfiles especializados}

El clustering K-means fue exitoso en segregar usuarios en tres categorías bien diferenciadas según su consumo de datos:

\begin{longtable}{@{}lcccc@{}}
\toprule
Perfil & Usuarios & Consumo Total & \% del Tráfico & Consumo/Usuario \\
\midrule
\endhead
\textbf{Bajo} & 459 & 25,020,000 MB & 29.51\% & $\sim$54,500 MB \\
\textbf{Normal} & 422 & 45,440,000 MB & 53.59\% & $\sim$107,700 MB \\
\textbf{Alto} & 1 & 14,340,000 MB & 16.91\% & 14,340,000 MB \\
\bottomrule
\end{longtable}

\textbf{Hallazgos principales:}
\begin{quote}
    \begin{itemize}[leftmargin=*]
        \item El perfil \textbf{Normal} domina tanto en cantidad de usuarios (45\%) como en volumen de tráfico (54\%).
        \item Los usuarios de perfil \textbf{Bajo} representan el 47\% de la base pero solo generan el 30\% del tráfico.
        \item Un único usuario de perfil \textbf{Alto} genera casi el 17\% de todo el tráfico, sugiriendo una posible anomalía o cliente empresarial.
        \item Estos tres segmentos justifican el desarrollo de planes y tratamientos especializados para cada grupo.
    \end{itemize}
\end{quote}

\subsubsection*{Objetivo 2: Patrones horarios por grupo}

El análisis del período 00:00--8:00 horas reveló patrones de actividad diferenciados y bien definidos:

\

\textbf{Comportamiento horario general:}
\begin{quote}
    \begin{itemize}
        \item \textbf{Hora 00:00} (Pico nocturno): 24,040,000 MB, dominado por Normal (62\%) y Bajo (38\%) -- 508 usuarios activos.
        \item \textbf{Hora 03:00} (Pico principal): 23,540,000 MB con dominio absoluto del perfil Alto (60\% del consumo).
        \item \textbf{Horas 06:00--07:00} (Declive): Consumo decrece a 2,300,000 MB y 269,000 MB; usuarios Bajo se vuelven dominantes (59--81\%).
    \end{itemize}
\end{quote}

\

\textbf{Patrones identificados por perfil:}
\begin{quote}
    \begin{enumerate}
        \item \textbf{Perfil Bajo}: Actividad sostenida en madrugada (00:00--04:00). Mayor concentración 04:00--07:00 (59\% del consumo horario). Usuarios consistentes pero con bajo volumen.
        \item \textbf{Perfil Normal}: Distribuido en dos picos: 00:00--03:00 con 62\% participación. Mantiene presencia significativa durante todo el periodo. Más estable en relación consumo/usuarios.
        \item \textbf{Perfil Alto}: Concentración extrema en hora punta 03:00 (60\% del consumo total). Ausente en otras horas. Sugiere usuario empresarial o con patrón de uso muy específico.
    \end{enumerate}
\end{quote}

\

\textbf{Oportunidades identificadas:}
\begin{quote}
    \begin{itemize}
        \item Se requiere mayor capacidad de infraestructura en horario 00:00--03:00.
        \item Es posible optimizar ancho de banda según perfiles: mayor capacidad para Normal, menor para Bajo en off-peak.
        \item El usuario Alto puede requerir línea dedicada especial en horas punta.
    \end{itemize}
\end{quote}

\subsubsection*{Objetivo 3: Análisis consumo vs tiempo de actividad}

Se analizó la viabilidad de proponer planes por tiempo de conexión en lugar de consumo de datos:

\begin{longtable}{@{}lccc@{}}
\toprule
Perfil & Duración Promedio & Consumo Promedio & Ratio (MB/hora) \\
\midrule
\endhead
\textbf{Bajo} & 2.37 h & 54,500 MB & 23,000 MB/h \\
\textbf{Normal} & 10.89 h & 107,700 MB & 9,900 MB/h \\
\textbf{Alto} & 2.31 h & 14,340,000 MB & 6,200,000 MB/h (anomalía) \\
\bottomrule
\end{longtable}

\textbf{Conclusión: SÍ es viable un modelo de planes por tiempo}

\begin{quote}
    \begin{itemize}
        \item \textbf{Para usuarios Bajo}: Presentan ratio más elevado pero consistente. Plan por tiempo sería muy beneficioso para evitar penalizaciones a quienes usan mucho en corto tiempo.
        \item \textbf{Para usuarios Normal}: Utilizan 10+ horas con consumo variable. Plan híbrido (horas + datos) sería óptimo.
        \item \textbf{Para usuario Alto}: Requiere plan empresarial especial.
    \end{itemize}
\end{quote}

\

\textbf{Recomendaciones operacionales:}
\begin{quote}
    \begin{itemize}
        \item Ofrecer planes diarios/mensuales por horas de conexión sería atractivo para usuarios como los del perfil Bajo.
        \item Combinar límites de tiempo con límites de datos mejora la experiencia del usuario Normal.
    \end{itemize}
\end{quote}

\subsubsection*{Objetivo 4: Preferencia de bajos consumidores por otros servicios}

Se realizó un estudio estadístico exhaustivo para validar si los bajos consumidores de datos tienen preferencia por otros servicios (SMS y Voz), utilizando test de hipótesis con 95\% de confianza.

\

\textbf{Metodología:}

\begin{quote}
    \begin{enumerate}
        \item \textbf{Fuente de datos:} Base de datos muestra.xlsx con 10,000 registros de múltiples servicios.
        \item \textbf{Clasificación de usuarios:} 
        \begin{itemize}[leftmargin=*]
            \item SERVICE\_CATEGORY=5: Datos (881 usuarios únicos).
            \item SERVICE\_CATEGORY=2: SMS (324 usuarios únicos).
            \item SERVICE\_CATEGORY=1: Voz (89 usuarios únicos).
        \end{itemize}
        \item \textbf{Segmentación:} Aplicación de K-means en cada servicio para clasificar usuarios en Bajo/Normal/Alto.
        \item \textbf{Análisis:} Test binomial unilateral con Intervalo de Confianza de Wilson al 95\%.
    \end{enumerate}
\end{quote}

\textbf{Hipótesis planteada:} ``60\% o más de los bajos consumidores de datos prefieren otro servicio'' (son altos/normales en SMS o Voz).

\paragraph*{Estudio 1: SMS - Preferencia de Mensajes}

\ 

\textbf{Datos del cruce:}
\begin{quote}
    \begin{itemize}
        \item Bajos consumidores de DATOS identificados: \textbf{152 usuarios}.
        \item Usuarios que son ALTO/NORMAL en SMS: \textbf{14 usuarios (9.21\%)}.
        \item Usuarios que son BAJO en SMS: \textbf{138 usuarios (90.79\%)}.
    \end{itemize}
\end{quote}

\

\textbf{Resultados estadísticos (95\% confianza):}
\begin{quote}
    \begin{itemize}
        \item p-value: 1.000000 (altamente no significativo).
        \item Intervalo de Confianza: [5.57\%, 14.87\%].
    \end{itemize}
\end{quote}

\

\textbf{Conclusión SMS:} HIPÓTESIS RECHAZADA.

\

\paragraph*{Estudio 2: Voz - Preferencia de Llamadas}

\

\textbf{Datos del cruce:}
\begin{quote}
    \begin{itemize}
        \item Bajos consumidores de DATOS con datos de VOZ: \textbf{43 usuarios}.
        \item Usuarios que son ALTO/NORMAL en VOZ: \textbf{9 usuarios (20.93\%)}.
        \item Usuarios que son BAJO en VOZ: \textbf{34 usuarios (79.07\%)}.
    \end{itemize}
\end{quote}

\

\textbf{Resultados estadísticos (95\% confianza):}
\begin{quote}
    \begin{itemize}
        \item p-value: 1.000000 (altamente no significativo).
        \item Intervalo de Confianza: [11.42\%, 35.21\%].
    \end{itemize}
\end{quote}

\

\textbf{Conclusión VOZ:} HIPÓTESIS RECHAZADA.

\

\textbf{CONCLUSIÓN GENERAL DEL OBJETIVO 4:}

La hipótesis es FALSA en ambos servicios.

Los bajos consumidores de datos \textbf{NO prefieren otros servicios}. Por el contrario:
\begin{itemize}
    \item Solo 9.21\% de bajos consumidores de datos son altos/normales en SMS.
    \item Solo 20.93\% de bajos consumidores de datos son altos/normales en Voz.
    \item Ambas proporciones están \textbf{lejos del 60\% hipotético}.
\end{itemize}

\

\textbf{Interpretación:} Existe un perfil de usuario claramente diferenciado:
\begin{itemize}
    \item Usuarios ``frugales'' que consumen poco en TODOS los servicios (datos, SMS, voz).
    \item No hay compensación de consumo entre servicios.
    \item Los bajos consumidores de datos mantienen un patrón consistente de bajo uso en toda la plataforma.
\end{itemize}

\

\textbf{Implicaciones para la estrategia comercial:}
\begin{itemize}
    \item Los bajos consumidores de datos NO son un segmento que busque alternativas en otros servicios.
    \item Estrategias de upsell basadas en ofrecer SMS/Voz a estos usuarios tendrían bajo potencial.
    \item Estos usuarios pueden estar orientados a aplicaciones OTT (WhatsApp, redes sociales) en lugar de servicios tradicionales.
    \item Se recomienda analizar patrones de uso de datos más profundamente (tipo de aplicación: streaming, redes sociales, etc.).
\end{itemize}

\subsubsection*{Objetivo 5: Evaluación y Optimización del Número de Clusters}

Se realizó un análisis exhaustivo para determinar el número óptimo de clusters (K) usando el \textbf{Coeficiente de Silueta}, una métrica que combina cohesión (cercanía dentro del cluster) y separación (distancia entre clusters).

\

\textbf{Metodología:}

El Coeficiente de Silueta se define como:
\[
\text{Silueta} = \frac{d_{\text{inter}} - d_{\text{intra}}}{\max(d_{\text{inter}}, d_{\text{intra}})}
\]

Donde:
\begin{itemize}
    \item $d_{\text{intra}}$: distancia promedio dentro del cluster (menor es mejor)
    \item $d_{\text{inter}}$: distancia promedio al cluster más cercano (mayor es mejor)
    \item Rango: -1 a +1 (valores > 0.5 indican clustering bien definido)
\end{itemize}

\

\textbf{Resultados del análisis de K = 2 a 10:}

\begin{longtable}{@{}lcl@{}}
\toprule
K & Silueta & Evaluación \\
\midrule
\endhead
2 & 0.4488 & Clustering débil \\
3 & 0.4586 & Clustering débil (K actual) \\
4 & 0.4901 & Excelente -- K rechazado por singletons \\
\textbf{5} & \textbf{0.4902} & \textbf{Óptimo -- Mejor silueta} \\
6 & 0.4197 & Comienza a degradarse \\
7--10 & < 0.43 & Degradación progresiva \\
\bottomrule
\end{longtable}

\

\textbf{Comparativa K=4 vs K=5 (decisión crítica):}

\begin{longtable}{@{}lcccc@{}}
\toprule
Métrica & K=4 & K=5 & Decisión \\
\midrule
\endhead
Silueta & 0.4901 & 0.4902 & Prácticamente idéntica ($\Delta$=0.0001) \\
Distribución & Balanceada & Más balanceada & K=4 gana por singletons \\
Singletons & 1 (0.1\%) & 2 (0.2\%) & K=4 es mejor \\
Interpretabilidad & Alta & Media & K=4 es más simple \\
\textbf{Selección} & \textbf{ELEGIDO} & Descartado & Mejor opción \\
\bottomrule
\end{longtable}

\

\textbf{Conclusión del análisis:}

\textbf{K=4 es el número óptimo de clusters}

\begin{itemize}
    \item Silueta es \textbf{6.9\% mejor que K=3}
    \item K=4 y K=5 tienen silueta prácticamente idéntica ($\Delta$=0.0001: 0.4901 vs 0.4902)
    \item \textbf{K=4 tiene mejor distribución:} 1 singleton vs 2 en K=5
    \item K=4 es más simple operacionalmente
\end{itemize}

\

\textbf{Nuevos perfiles optimizados (K=4):}

\begin{longtable}{@{}lccc@{}}
\toprule
Perfil & Usuarios & \% del Total & Características \\
\midrule
\endhead
\textbf{Cluster 0} & 396 & 44.9\% & Usuarios normales \\
\textbf{Cluster 1} & 444 & 50.3\% & Usuarios bajos-medio \\
\textbf{Cluster 2} & 41 & 4.6\% & Usuarios especializados \\
\textbf{Cluster 3} & 1 & 0.1\% & Outlier empresarial \\
\bottomrule
\end{longtable}

\

\textbf{Beneficios operacionales de K=4:}

\begin{enumerate}
    \item \textbf{Mejor balance:} 1 singleton vs 2 en K=5.
    \item \textbf{Silueta prácticamente idéntica:} 0.4901 vs 0.4902 (diferencia de 0.0001).
    \item \textbf{Calidad mejorada:} 6.9\% superior a K=3.
    \item \textbf{Simplicidad:} 4 clusters más fáciles de gestionar operacionalmente.
\end{enumerate}

\

\textbf{Recomendación:}

Se recomienda adoptar \textbf{K=4 clusters} para futuras segmentaciones de usuarios, reemplazando el K=3 anterior. Aunque K=5 tiene silueta marginalmente mejor (0.0001 de diferencia), K=4 ofrece mejor distribución con menos singletons y es más simple de interpretar operacionalmente.

\end{document}